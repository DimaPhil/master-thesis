%% Задание
%%% Техническое задание и исходные данные к работе
\technicalspec{По данному неориентированному невзвешенному графу социальной сети и выделенным в нем вершинам-запросам требуется найти плотный подграф, содержащий все или большинство данных вершин, то есть так называемое сообщество социальной сети.}

%%% Содержание выпускной квалификационной работы (перечень подлежащих разработке вопросов)
\plannedcontents{Пояснительная записка должна демонстрировать новый подход к решению этой задачи, а также его плюсы и минусы по сравнению с предыдущими методами. Должно быть произведено сравнение оптимальности ответов, времени работы, а также других метрик для всех рассмотренных и приведенных алгоритмов.}

%%% Исходные материалы и пособия 
\plannedsources{\begin{enumerate}
    \item Faloutsos C., McCurley K.S. and Tomkins A. Fast discovery of connection subgraphs;
    \item Faloutsos C., Tong H. Center-Piece Subgraphs: Problem Definition and Fast Solutions;
    \item Ruchansky N. et al. The Minimum Wiener Connector Problem;
    \item Gionis A., Mathioudakis M. and Ukkonen A. Bump hunting in the dark: Local discrepancy maximization on graphs.
\end{enumerate}}

%%% Календарный план
\addstage{Ознакомление с исходными статьями}{10.2016}
\addstage{Поиск новых статей}{11.2016}
\addstage{Изучение новых статей, выбор бейзлайна}{12.2016}
\addstage{Реализация бейзлайна}{03.2017}
\addstage{Исследование темы, предложения улучшений бейзлайна}{03.2017}
\addstage{Реализация улучшений, сравнение практических результатов с теоретическими}{04.2017}
\addstage{Написание пояснительной записки}{05.2017}

%%% Цель исследования
\researchaim{В рамках работы необходить предложить улучшение существующих методов поиска сообщества по выделенным в графе социальной сети вершинам, предложить алгоритм отсеивания шума в запросе для получения более плотного подграфа в качестве ответа.}

%%% Задачи, решаемые в ВКР
\researchtargets{\begin{enumerate}
    \item Провести исследование описанной задачи;
    \item Выделить одно или несколько базовых решений;
    \item Реализовать выбранные базовые решения и сравнить их результаты с результатами с статьях;
    \item Разработать методы улучшения базовых решений, способные учитывать шум в запросах;
    \item Реализовать разработанные методы и сравнить полученные результаты с результатами базовых решений.
\end{enumerate}}

%%% Использование современных пакетов компьютерных программ и технологий
\advancedtechnologyusage{Для реализации алгоритмов был использован язык программирования \textit{Java 1.8}. Также был использован фреймворк \textit{Kryo} для быстрой и автоматической сериализации и десериализации большого объема данных. Для вычислений, использующих большой объем памяти, были использованы технологии \textit{ssh} и \textit{slurm} для подключения и работы на серверах кластера Универстита ИТМО, имеющих $128$, $256$ и $496$ Гб оперативной памяти.}

%%% Краткая характеристика полученных результатов 
\researchsummary{Результаты, полученные в статье, показывают, что на запросах, содержащих шум, предложенное решение работает оптимальнее всех предыдущих. На запросах без шума решение работает не хуже, а иногда даже лучше существующих решений.}

%%% Гранты, полученные при выполнении работы 
\researchfunding{}

%%% Наличие публикаций и выступлений на конференциях по теме выпускной работы
\researchpublications{}
