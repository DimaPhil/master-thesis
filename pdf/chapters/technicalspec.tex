%% Задание
%%% Техническое задание и исходные данные к работе
\technicalspec{По данному атрибутированному графу социальной сети, выделенным в нем множеству вершин $V_q$ и множеству атрибутов $W_q$ требуется найти плотный подграф, содержащий все или почти все вершины из $V_q$, атрибуты чьих вершин хорошо коррелируют с $W_q$. Этот подграф называется сообществом социальной сети.}

%%% Содержание выпускной квалификационной работы (перечень подлежащих разработке вопросов)
\plannedcontents{Пояснительная записка должна демонстрировать более оптимальный подход к решению к поставленной задачи, чем существующим. Должно быть произведено сравнение нового алгоритма и текущих решений, выделены плюсы и минусы нового алгоритма. В рассмотрение должны браться значения различных метрик, таких как плотности подграфа, корреляция атрибутов, время работы алгоритма, и т.п.}

%%% Исходные материалы и пособия 
\plannedsources{\begin{enumerate}
    \item Faloutsos C., McCurley K.S. and Tomkins A. Fast discovery of connection subgraphs;
    \item Faloutsos C., Tong H. Center-Piece Subgraphs: Problem Definition and Fast Solutions;
    \item Gionis A., Mathioudakis M. and Ukkonen A. Bump hunting in the dark: Local discrepancy maximization on graphs.
    \item Barbieri N., Bonchi F., Galimberti E., Gullo F. Efficient and effective community search.
\end{enumerate}}

%%% Календарный план
\addstage{Ознакомление и анализ исходных статей}{10.2017}
\addstage{Проведение поиска и анализа новый статей}{02.2018}
\addstage{Обсуждение алгоритмов, выбор одного или нескольких бейзлайнов}{03.2018}
\addstage{Реализация бейзлайнов}{07.2018}
\addstage{Обсуждение полученных результатов, предложение улучшений}{08.2018}
\addstage{Последовательные реализации улучшений, предложение новых}{03.2019}
\addstage{Написание пояснительной записки}{05.2019}

%%% Цель исследования
\researchaim{В рамках работы необходить предложить улучшение существующих методов поиска атрибутированного сообщества по выделенным в графе социальной сети вершинам и множеству атрибутов.}

%%% Задачи, решаемые в ВКР
\researchtargets{\begin{enumerate}
    \item Провести подробное исследование поставленной задачи;
    \item Выбрать одно или несколько базовых решений~--- бейзлайнов;
    \item Реализовать выбранные бейзлайны и сравнить их результаты с результатами из соответствующих статей;
    \item Разработать методы улучшения базовых решений;
    \item Реализовать разработанные методы и сравнить полученные результаты с результатами базовых решений;
    \item Сравнить итоговые результаты с результатами исходных статей.
\end{enumerate}}

%%% Использование современных пакетов компьютерных программ и технологий
\advancedtechnologyusage{Для реализации алгоритмов был использован язык программирования \textit{Java 1.11}. Для вычислений, использующих большой объем памяти, были использованы технологии \textit{ssh} и \textit{slurm} для подключения и работы на серверах кластера Универстита ИТМО, имеющих $128$, $256$ и $496$ Гб оперативной памяти соответственно.}

%%% Краткая характеристика полученных результатов 
\researchsummary{Результаты, полученные в статье, показывают, что метрики плотности итогового подграфа и корреляции атрибутов увеличились по сравнению с предыдущими алгоритмами.}

%%% Гранты, полученные при выполнении работы 
\researchfunding{}

%%% Наличие публикаций и выступлений на конференциях по теме выпускной работы
\researchpublications{}
