%% Макрос для введения. Совместим со старым стилевиком.
\startprefacepage

%\section{Краткое описание}

Последнее время изучение и аналих социальных сетей представляет большой интерес. 
Один из примеров анализа социальных сетей~--- нахождение сообществ пользователей. 
Многие работы исследуют сообщества целого графа, в то время как большой интерес также представляет анализ сообществ, 
образованных только данным множеством вершин~--- по данным выделенным вершинам в социальном графе найти плотный подграф, содержащий их все. 
Эта задача также широко изучена, однако наложение условия на наличие всех вершин в итоговом подграфе не всегда оптимально для нахождения наиболее плотного подграфа, так как этот подход не учитывает возможный <<шум>> в запросе.

В этой работе представлен метод оптимизации текущих методов решения описанной задачи при условии возможного шума в запросе. Метод показал хорошие результаты на реальных графах социальных сетей, а также улучшил результаты предыдущих работ.

%\section{Актуальность}

Актуальность исходной задачи (требующей наличие всех выделенных вершин в итоговом подграфе) проявляется во многих областях:
\begin{enumerate}
  \item Полиция~--- зная нескольких подозреваемых, выяснить, кто еще мог быть соучастником преступления, участником банды или группировки;
  \item Социальные сети~--- после добавления одного или нескольких друзей в социальной сети, также предлагается еще несколько, которые тесно связаны с недавно добавленными и которых вы вероятно всего тоже знаете;
  \item Медицина~--- по нескольким заболевшим определить других наиболее вероятно инфицированных, используя социальный граф связей и знакомств;
  \item Организация мероприятий~--- если на важное мероприятие требуется позвать несколько спикеров, также понять, кого еще, тесно связанного с этими людьми, хорошо было бы увидеть на этом мероприятии.
\end{enumerate}

Актуальность нашей задачи (требующей наличие не всех, а только большинства выделенных вершин в итоговом подграфе) также проявлется в этих областях, причем, как можно заметить, наша задача более актуальна в реальной жизни:

\begin{enumerate}
  \item Полиция~--- определить остальных соучастников преступления, участников банды или группировки, учитывая, что некоторые подозреваемые могли быть взяты только для этого дела и к группировке отношения не имеют;
  \item Социальные сети~--- рекомендация друзей после недавнего добавления нескольких людей, учитывая то, что добавленные друзья могут быть из разных социальных сообществ и связь между ними может быть только через вас;
  \item Медицина~--- определить наиболее вероятно инфицированных, учитывая тот фактор, что некоторые результаты могли оказаться ложно-положительными;
  \item Организация мероприятий~--- приглашение людей, наиболее связанных со спикерами, учитывая то, что спикеры могут быть слабо связаны друг с другом.
\end{enumerate}

%\section{Цель}

%Цель этой бакалаврской работы~--- изучить поставленную задачу и ее текущие методы решения, предложить более эффективный метод решения задачи.

%\section{Новизна}

Практически все существующие по этой теме статьи рассматривают задачу поиска подграфа, содержащего все выделенные вершины. Наш же алгоритм вводит возможность взятия в итоговое подмножество не всех выделенных вершин, а только большинства из них, и показывает результаты лучше, чем предыдущие алгоритмы на запросах, содержащих шум.

%\section{Структура работы}

В главе $1$ приведен обзор темы: введены основные определения, используемые в статье и необходимые для понимания остальных терминов, приведен обзор существующих решений поставленной задачи с кратким описанием каждого решения. В конце главы уточнены требования к работе исходя из изложенной информации.

В главе $2$ предложено подробное описание алгоритма, разбитое на фазы. В главе также приведены примеры работы алгоритма на небольших запросах с подробным описанием и иллюстрациями.

В главе $3$ приведено описание проведенных экспериментов на реальных данных~--- какие данные были выбраны для анализа, как строились и проводились эксперименты. Также приведены диаграммы и таблицы, описывающие результаты экспериментов, показывающие и доказывающие улучшения, описанные в этой работе.
